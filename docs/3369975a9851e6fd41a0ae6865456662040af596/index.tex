% Options for packages loaded elsewhere
\PassOptionsToPackage{unicode}{hyperref}
\PassOptionsToPackage{hyphens}{url}
%
\documentclass[
]{article}
\usepackage{amsmath,amssymb}
\usepackage{iftex}
\ifPDFTeX
  \usepackage[T1]{fontenc}
  \usepackage[utf8]{inputenc}
  \usepackage{textcomp} % provide euro and other symbols
\else % if luatex or xetex
  \usepackage{unicode-math} % this also loads fontspec
  \defaultfontfeatures{Scale=MatchLowercase}
  \defaultfontfeatures[\rmfamily]{Ligatures=TeX,Scale=1}
\fi
\usepackage{lmodern}
\ifPDFTeX\else
  % xetex/luatex font selection
\fi
% Use upquote if available, for straight quotes in verbatim environments
\IfFileExists{upquote.sty}{\usepackage{upquote}}{}
\IfFileExists{microtype.sty}{% use microtype if available
  \usepackage[]{microtype}
  \UseMicrotypeSet[protrusion]{basicmath} % disable protrusion for tt fonts
}{}
\makeatletter
\@ifundefined{KOMAClassName}{% if non-KOMA class
  \IfFileExists{parskip.sty}{%
    \usepackage{parskip}
  }{% else
    \setlength{\parindent}{0pt}
    \setlength{\parskip}{6pt plus 2pt minus 1pt}}
}{% if KOMA class
  \KOMAoptions{parskip=half}}
\makeatother
\usepackage{xcolor}
\usepackage[margin=1in]{geometry}
\usepackage{longtable,booktabs,array}
\usepackage{calc} % for calculating minipage widths
% Correct order of tables after \paragraph or \subparagraph
\usepackage{etoolbox}
\makeatletter
\patchcmd\longtable{\par}{\if@noskipsec\mbox{}\fi\par}{}{}
\makeatother
% Allow footnotes in longtable head/foot
\IfFileExists{footnotehyper.sty}{\usepackage{footnotehyper}}{\usepackage{footnote}}
\makesavenoteenv{longtable}
\usepackage{graphicx}
\makeatletter
\def\maxwidth{\ifdim\Gin@nat@width>\linewidth\linewidth\else\Gin@nat@width\fi}
\def\maxheight{\ifdim\Gin@nat@height>\textheight\textheight\else\Gin@nat@height\fi}
\makeatother
% Scale images if necessary, so that they will not overflow the page
% margins by default, and it is still possible to overwrite the defaults
% using explicit options in \includegraphics[width, height, ...]{}
\setkeys{Gin}{width=\maxwidth,height=\maxheight,keepaspectratio}
% Set default figure placement to htbp
\makeatletter
\def\fps@figure{htbp}
\makeatother
\setlength{\emergencystretch}{3em} % prevent overfull lines
\providecommand{\tightlist}{%
  \setlength{\itemsep}{0pt}\setlength{\parskip}{0pt}}
\setcounter{secnumdepth}{-\maxdimen} % remove section numbering
\usepackage{booktabs}
\usepackage{longtable}
\usepackage{array}
\usepackage{multirow}
\usepackage{wrapfig}
\usepackage{float}
\usepackage{colortbl}
\usepackage{pdflscape}
\usepackage{tabu}
\usepackage{threeparttable}
\usepackage{threeparttablex}
\usepackage[normalem]{ulem}
\usepackage{makecell}
\usepackage{xcolor}
\ifLuaTeX
  \usepackage{selnolig}  % disable illegal ligatures
\fi
\usepackage{bookmark}
\IfFileExists{xurl.sty}{\usepackage{xurl}}{} % add URL line breaks if available
\urlstyle{same}
\hypersetup{
  pdftitle={Informing Management of B.C. Hand-Harvested Invertebrates: Operating Models, MPs and MSE},
  pdfauthor={Tom Carruthers tom@bluematterscience.com},
  hidelinks,
  pdfcreator={LaTeX via pandoc}}

\title{Informing Management of B.C. Hand-Harvested Invertebrates:
Operating Models, MPs and MSE}
\author{Tom Carruthers
\href{mailto:tom@bluematterscience.com}{\nolinkurl{tom@bluematterscience.com}}}
\date{16 March, 2025}

\begin{document}
\maketitle

~

\begin{center}\rule{0.5\linewidth}{0.5pt}\end{center}

\begin{center}\rule{0.5\linewidth}{0.5pt}\end{center}

~

\begin{center}\rule{0.5\linewidth}{0.5pt}\end{center}

\subsubsection{Disclaimer}\label{disclaimer}

The following work is preliminary and intended only as tool for
eliciting feedback on data, modelling and other aspects of these
fisheries.

None of these results are final.

These analyses do not necessarily reflect the point of view of DFO or
other funders and in no way anticipate DFO future policy in this area.

\begin{center}\rule{0.5\linewidth}{0.5pt}\end{center}

\subsubsection{Objective}\label{objective}

Establish operating models for at least four species of hand-harvested
invertebrates in B.C. for the purposes of informing management decision
making including data collection, suitable stock assessment approaches,
reference points and harvest control rules.

\begin{center}\rule{0.5\linewidth}{0.5pt}\end{center}

\subsubsection{Project details}\label{project-details}

\begin{longtable}[]{@{}ll@{}}
\toprule\noalign{}
\endhead
\bottomrule\noalign{}
\endlastfoot
Term & April 2022 - March 2023, April 2023 - March 2024, April 2024 -
March 2025 \\
Funding body & Canadian Department of Fisheries and Oceans (DFO) \\
Funding stream & ProServices, Medium Complexity Bid \\
Solicitation No. & 30003600, 30004307 \\
Contract No. & 4500038008, 4500051010, 4600000482 \\
Project Partners & Blue Matter Science Ltd. \\
Blue Matter Team & Tom Carruthers, Adrian Hordyk, Quang Huynh \\
DFO Principal Investigators & Shannon Obradovich, Mackenzie Mazur \\
\end{longtable}

\begin{center}\rule{0.5\linewidth}{0.5pt}\end{center}

\subsubsection{Operating models}\label{operating-models}

An operating model is a theoretical description of fishery and
population dynamics used for the testing of management strategies that
could include, for example, data collection protocols, stock assessment
methods, harvest control rules, enforcement policies and reference
points. In fisheries, operating models are used in closed-loop
simulation to test management procedures (aka. harvest strategy)
accounting for feedback among the system, data, management procedure and
implementation. A management procedure is a rule that calculates
management advice from data. Management Strategy Evaluation uses
closed-loop simulation of management procedures as a core technical
component but is a wider process of stakeholder and manager engagement
that identifies system uncertainties, performance metrics, viable
management procedures, ultimately aiming to adopt an MP for the
provision of management advice for an established time period.

~

\paragraph{Reference Case Operating
Models}\label{reference-case-operating-models}

The reference case operating model is used as the single `base'
operating model from which reference set and robustness set operating
models are specified. Reference and robustness tests are typically
1-factor departures from the reference case OM, however sometimes
reference set OMs are organized in a factorial grid across primary axes
of uncertainty.

~

\paragraph{Reference Set Operating
Models}\label{reference-set-operating-models}

Reference set operating models span a plausible range of the core
uncertainties for states of nature. These are often the types of
alternative parameterizations or assumptions that would be included in a
stock assessment sensitivity analysis.

The role of the reference set operating models is to provide the central
basis for evaluating the performance of candidate management procedures,
for example rejecting badly performing harvest strategies.

~

\paragraph{Robustness Set Operating
Models}\label{robustness-set-operating-models}

Robustness set operating models are intended to include additional
sources of uncertainty for providing further discrimination among
management procedures that perform comparably among reference set
operating models.

Robustness operating models often represent system states of nature that
are not empirically informed or are hypotheses of a subset of
stakeholders.

~

\begin{center}\rule{0.5\linewidth}{0.5pt}\end{center}

\subsubsection{\texorpdfstring{Geoduck (\emph{Panopea
generosa})}{Geoduck (Panopea generosa)}}\label{geoduck-panopea-generosa}

\paragraph{Operating Model
Specification}\label{operating-model-specification}

Geoduck operating models were constructed assuming that discrete
populations occur at the resolution of statistical area (management
area). Models were conditioned using the Rapid Conditioning Model (RCM)
of openMSE (SAMtool package, Huynh et al.~2023) and fitted to historical
catches, standardized catch-per-unit-effort, sub area age composition
data, a current estimate of absolute biomass and biomass trends within
statistical area based on bed-level survey data. Given an assumption of
asymptotic fleet selectivity and the availability of the absolute
biomass estimate, it was possible to estimate natural mortality rate
from an informative prior.

The Reference Case operating model presented here is for statistical
area 14 which had numerous age-composition data.

MSE-style closed-loop projections were undertaken for the current
harvest rate (the principal management guideline) and current catch
levels.

~

Figure 1. Statistical Areas for which age data were available and RCM
operating models were fitted.

~

\paragraph{Reference Case Operating
Model}\label{reference-case-operating-model}

\href{OM_Descriptions/Ref_Case_Geoduck_Annotated.html}{Reference Case
Operating Model Description (.html)}

~

\paragraph{Comparison of Stat Area Operating
Models}\label{comparison-of-stat-area-operating-models}

Across the 22 Stat. areas for which age-data were available and models
could be conditioned to data

\href{Comparisons/Geoduck_Area_Comparison.html}{Stat Area Comparisons
(.html)}

~

Table 1a. Geoduck RCM model fits

~

\paragraph{Custom Analyses}\label{custom-analyses}

A summary of the 2024 custom analyses for all species is available
\href{Project_Info/Custom\%20Analyses\%202023-2024.pdf}{here}

\href{MSEs/Geoduck_Demo_MSE.html}{Some initial closed-loop MSE-type
projections were also conducted}

~

\paragraph{Project Status}\label{project-status}

~

Table 1b. Project Updates and Progress

\begin{longtable}[]{@{}ll@{}}
\toprule\noalign{}
Update & Details \\
\midrule\noalign{}
\endhead
\bottomrule\noalign{}
\endlastfoot
Hierarchical growth model & M. Burton (DFO) developed a hierarchical
model in TMB / stan for estimating Stat. area growth parameters. MCMC
samples are now used in the operating models \\
Current absolute biomass & Data were updated to 2022 and current
absolute biomass estimates (only absolute index) are now used to inform
the model \\
Bed-level trends & Bed-level biomass survey time series are used to
inform the trend in recent biomass. This includes up to 10 individual
bed indices or those that encompass 80\% of surveyed biomass (calculated
by the mean value per bed over all years) \\
Standardized catch per unit effort & Dr M. Mazur (DFO) developed a
generalized linear model to standardize commercial catch data according
to various variates. The model included temporal autocorrelation to
prevent the estimation of large, implausible differences in inferred
abundance. \\
Selectivity parameterized by age & After around age 10, geoduck reach
close to their asymptotic length. Parameterizing operating models
according to age is necessary to fit the age composition data. \\
Age composition disaggregated & In order to account for regional
variation in age structure, the age composition data are disaggregated
by subarea. All are assigned dome-shaped selectivity except the sub area
with the highest mean age which is assigned logistic `flat-topped'
selectivity. \\
Natural mortality prior & Given the absolute biomass survey and age
composition data it is possible to estimate natural mortality. This is
assigned an informative prior of mean 0.05 with a CV of 15\%. \\
Lognormal likelihood for age data & The model is now fitted to age
composition data with a lognormal likelhood function - this reduces the
tendency for the model to estimate occasional high recruitments from a
single year of data. \\
Truncated age composition data & Only age composition data after 1990
are used in operating model conditioning \\
\end{longtable}

~

Table 1c. Geoduck assumptions and to-do list

\begin{longtable}[]{@{}ll@{}}
\toprule\noalign{}
Assumptions & To Do \\
\midrule\noalign{}
\endhead
\bottomrule\noalign{}
\endlastfoot
Stat. Area is the biological unit & Make generic management performance
outputs \\
Commercial Catch CV of 5\% & Add correct coords to OM objects by stat
area \\
Annual age data effective sample size of 40 & Make a coastwide OM by
aggregating data \\
5\% Selectivity at 100mm, full selectivity at 120mm (all Stat Areas) &
Test open-closure rules \\
Informative M prior of 0.05 with CV of 15\% & Work on calculation of
coast-wide LRP \\
Absolute biomass estimates are only final historical year & Test
efficacy of small-scale spatial closures \\
Somatic Growth follows a von-B growth equation & Robustness (M,
unreported catches, somatic growth, rec strength) \\
Maturity is from 2003 study & \\
\end{longtable}

~

\paragraph{Geoduck Meeting Notes etc}\label{geoduck-meeting-notes-etc}

\href{Project_Info/2023_11_17_CSRF_Geoduck_Review.pdf}{2023 Meeting
Notes (.pdf)}

~

\begin{center}\rule{0.5\linewidth}{0.5pt}\end{center}

\subsubsection{\texorpdfstring{Manila Clam (\emph{Venerupis
philippinarum})}{Manila Clam (Venerupis philippinarum)}}\label{manila-clam-venerupis-philippinarum}

\paragraph{Operating Model
Specification}\label{operating-model-specification-1}

Manila clam operating were conditioned to historical catches from 1999
onwards. The model assumes an equilibrium annual catch equal to 75\% of
the mean catch from 1999-2003.

The model is fitted to a CMA-level annual time series of survey
densities. This was calculated by aggregating density estimates from
beaches, where the beach-level data include linear interpolation among
years with observed data, and constant extrapolation before and after
the first and last observed years, respectively.

The model was also fitted to fishery length composition data and survey
age composition (annulus data). In both cases the selectivities were
assumed to be logistic. The fishery selectivity function was
parameterized by length, the survey selectivity was parameterized by
age. Somatic growth and weight-length parameters were sampled from a
multivariate normal distribution arising from variance-covariance matrix
of parameter estimates from an MLE fit to annulus data by Statistical
Area.

Since the available data only very weakly inform depletion, the models
are configured with an additional prior on current stock depletion for
testing robustness of management options to varying levels of stock
status.

The operating models were fitted using the
\href{https://samtool.openmse.com/reference/RCM.html}{Rapid Conditioning
Model (RCM)} included in the openMSE framework.

~

Figure 2. Location of Clam Management Areas.

~

\paragraph{Comparison of Operating Models Across
CMAs}\label{comparison-of-operating-models-across-cmas}

Across the 4 Clam Management Areas for which age-data were available and
models could be conditioned to data

\href{Comparisons/Manila_Clam_Area_Comparison.html}{Stat Area
Comparisons (.html)}

~

Table 2a. Manila Clam RCM model fits

~

\paragraph{Reference Case Operating
Model}\label{reference-case-operating-model-1}

\href{OM_Descriptions/Ref_Case_Manila_Clam_Annotated.html}{Reference
Case Operating Model Description (.html)}

~

\paragraph{Custom Analyses}\label{custom-analyses-1}

An investigation of minimum size limits and rebuilding was conducted and
is documented
\href{Project_Info/Custom\%20Analyses\%202023-2024.pdf}{here}

\href{MSEs/Manila_Clam_Demo_MSE.html}{Example closed-loop MSE-type
projections were also conducted (.html)}

~

\paragraph{Project Status}\label{project-status-1}

Table 2b. Project Updates and Progress

\begin{longtable}[]{@{}ll@{}}
\toprule\noalign{}
Update & Details \\
\midrule\noalign{}
\endhead
\bottomrule\noalign{}
\endlastfoot
Goes to 2022 & All data sources updated to 2022 \\
Includes IFMP survey data & Most recent data points were included which
allow for use of the index in simulated projections (as a basis for
responsive management) \\
Age composition data & From annulus data, now included and assumed to be
linked to the survey index series. \\
Placeholder for fishery effort data & Model can be configured to include
a metric of exploitation rate based on slip data (effort days). \\
\end{longtable}

~

Table 2c. Manila clam assumptions and to-do list

\begin{longtable}[]{@{}ll@{}}
\toprule\noalign{}
Assumptions & To Do \\
\midrule\noalign{}
\endhead
\bottomrule\noalign{}
\endlastfoot
Logistic (length - based) selectivity for the fleet & Investigate use of
effort slip data (managers). \\
Logistic (age-based) selectivity for the survey & Investigate
sensitivity to use of somatic growth derived from biological sampling
(as opposed to annulus data) \\
Equilibrium catches are 75\% the average of 1999-2003 & Sensitivity
alternative assumptions about `spool-up' equilibrium catches. \\
& Demonstrate robustness of current size limit management (much greater
than size at maturity). \\
& Sketch historical catch patterns (expert judgment) \\
& Tactical management options at the beach-level. E.g. open/closure
rules based on density \\
& Robustness to winter-kill / summar-kill events \\
& Develop OMs at varying spatial scales (e.g.~beach) \\
& Investigate possible uses of ICMP and its requirements (precision
etc) \\
\end{longtable}

~

\paragraph{Manilla Clam Meeting Notes
etc}\label{manilla-clam-meeting-notes-etc}

\href{Project_Info/2023_11_20_CSRFManilaClam_Review.pdf}{2023 Meeting
Notes (.pdf)}

~

\begin{center}\rule{0.5\linewidth}{0.5pt}\end{center}

\subsubsection{\texorpdfstring{Green Sea Urchin
(\emph{Strongylocentrotus
droebachiensis})}{Green Sea Urchin (Strongylocentrotus droebachiensis)}}\label{green-sea-urchin-strongylocentrotus-droebachiensis}

\paragraph{Operating Model
Specification}\label{operating-model-specification-2}

Green Urchin operating models were constructed assuming that discrete
populations occur at the resolution of Statistical Area (Management
Area). Models were conditioned using RCM and fitted to historical
catches, historical nominal catch-per-unit-effort, a survey relative
abundance index and fleet and survey length composition data.

The Reference Case operating model presented here is for Statistical
Area 12 which has numerous data and corresponds with a reasonably large
harvest of urchin.

MSE-style closed-loop projections were undertaken for the current
harvest rate (the principal management guideline) and current catch
levels.

~

Figure 3. Statistical Areas for which composition data and recent
catches were available and RCM operating models were fitted.

~

\paragraph{Comparison of Stat Area Operating
Models}\label{comparison-of-stat-area-operating-models-1}

\href{Comparisons/Green_Urchin_Area_Comparison.html}{Stat Area
Comparisons (.html)}

~

Table 3a. Green Urchin RCM model fits

~

\paragraph{Reference Case Operating
Model}\label{reference-case-operating-model-2}

\href{OM_Descriptions/Ref_Case_Green_Urchin_Annotated.html}{Reference
Case Operating Model Description (.html)}

~

\paragraph{Custom Analyses}\label{custom-analyses-2}

The results of demonstration analyses is documented
\href{Project_Info/Custom\%20Analyses\%202023-2024.pdf}{here}

\href{MSEs/Green_Urchin_Demo_MSE.html}{Also conducted were some example
closed-loop MSE-type projections (.html)}

~

\paragraph{Project Status}\label{project-status-2}

Table 3b. Project Updates and Progress

\begin{longtable}[]{@{}ll@{}}
\toprule\noalign{}
Update & Details \\
\midrule\noalign{}
\endhead
\bottomrule\noalign{}
\endlastfoot
Catches incorporated to inform model scale & Models updated to include
catch history \\
Model includes nominal CPUE & This is a placeholder for standardized
CPUE in a revision of the models \\
Multispecies survey density included & To allow for the testing of
density-based management procedures linked to simulated density \\
Survey relative biomass in legal / sublegal size classes & Two
additional indices included that are for legal and sublegal size classes
providing information about the population structure and potentially
allowing for management procedures that can account for shifts in size
composition. \\
Size composition data included for both fleet and survey & Two sources
of size composition data were combined in model conditioning for Stat.
areas 12 and 19 that are relatively `data rich' \\
Evaluation of alternative spatial definitions & Stat. areas were
aggregated into northern and southern areas to evaluate impact on
estimates of reference points and stock status. \\
\end{longtable}

~

Table 3c. Green urchin assumptions and to-do list

\begin{longtable}[]{@{}ll@{}}
\toprule\noalign{}
Assumptions & To Do \\
\midrule\noalign{}
\endhead
\bottomrule\noalign{}
\endlastfoot
Stat. Area is the biological unit & Get biomass estimates from Lyanne -
relative abundance indices? \\
Similar diameter-wet weight relationship among Stat areas & How to
construct a CPUE index? \\
& Somatic growth parameter ranges? Currently K comes from red urchin in
california and Linf from eye-balling the W-L data \\
& Distribution for L50 needed (or suitable range) \\
& Check Harvest header key \\
& Add correct coords to OM objects by stat area \\
& Robustness of size limit to changes in predators \\
& Opening / closures rules based on abundance. \\
\end{longtable}

~

\paragraph{Green Sea Urchin Meeting Notes
etc.}\label{green-sea-urchin-meeting-notes-etc.}

\href{Project_Info/Draft\%20summary\%20of\%20Urching\%20CSRF\%20Meeting_Feb13,\%202024_\%20Live\%20Notes.pdf}{Feb
2024 Meeting Notes (.pdf)}

~

\begin{center}\rule{0.5\linewidth}{0.5pt}\end{center}

\subsubsection{\texorpdfstring{Giant Red Sea Cucumber
(\emph{Apostichopus
californicus})}{Giant Red Sea Cucumber (Apostichopus californicus)}}\label{giant-red-sea-cucumber-apostichopus-californicus}

\paragraph{Operating Model
Specification}\label{operating-model-specification-3}

Sea cucumber operating models were configured to be numbers-based due to
the inability to age sea cucumbers and representatively measure/weigh
them. This entails the following working assumptions: knife-edge growth
to size / weight 1 at age 2; Recruitment based on a Beverton-Holt S-R
relationship calculated from mature numbers (the model includes a
fecundity growth parameter k where fecundity follows a cubic
relationship); Asymptotic selectivity from a young age class (e.g.~age
4-7).

Since harvesting is by diver and highly selective, regulation is by
harvest rate and bag-limit, the numbers-based operating model is
potentially appropriate for the management regime but requires realistic
modelling of the bag-limit impacts on discard rate and size selectivity.

Models were conditioned using RCM and fitted to historical catch in
numbers, an estimate of absolute numbers, and a time series of survey
density estimates.

~

Figure 4. Location of historical Sea Cucumber harvests.

~

\paragraph{Reference Case Operating
Model}\label{reference-case-operating-model-3}

\href{OM_Descriptions/Ref_Case_Sea_Cucumber_Annotated.html}{Reference
Case Operating Model Description (.html)}

~

\paragraph{Area comparisons among Subareas (revised for 2024 from
QMA)}\label{area-comparisons-among-subareas-revised-for-2024-from-qma}

\href{Comparisons/Sea_Cucumber_Area_Comparison.html}{Subarea Comparisons
(.html)}

~

Table 4a. Sea cucumber RCM model fits

~

\paragraph{Custom Analyses}\label{custom-analyses-3}

In 2023, MSE-style closed-loop projections were undertaken for the
current harvest rate, current catch levels and six example management
procedures that aim for 2.2, 4.2 and 6.7\% harvest rates without
alternating closures (C\_22, C\_42, C\_67) and also with alternating
closures (Alt\_22, Alt\_42, Alt\_67) whereby the QMA is closed and
opened every other projection year.

The results of these demonstration analyses is documented
\href{Project_Info/Custom\%20Analyses\%202023-2024.pdf}{here}

~

\paragraph{Project Status}\label{project-status-3}

Table 4b. Project Updates and Progress

\begin{longtable}[]{@{}ll@{}}
\toprule\noalign{}
Update & Details \\
\midrule\noalign{}
\endhead
\bottomrule\noalign{}
\endlastfoot
Revised to subarea models & Spatial unit changed back to subarea to
allow for complete catch, density and biomass reporting for each
model. \\
Top 10 by landings & Ten operating models were configured based on the
ten subareas with the most contributory historical landings \\
Density included for MPs & By including observations of survey density
it is now possible to configure management procedures and
opening/closing rules based on simulated density observations \\
Development of responsive management procedures & Allows for the testing
of adaptable harvest rate policies and open/closure rules \\
Rotational closure management procedures & Allows for evaluation of risk
equivalency among harvest rates with and without rotational closures \\
\end{longtable}

~

Table 4c. Green urchin assumptions and to-do list

\begin{longtable}[]{@{}ll@{}}
\toprule\noalign{}
Assumptions & To Do \\
\midrule\noalign{}
\endhead
\bottomrule\noalign{}
\endlastfoot
Subarea is the biological unit & Managers / scientists to select a range
of sub areas that span a range of exploitation, location and
environmental conditions \\
Modedl is numbers based (somatic growth, maturity, length-weight are
unknown and variable) rather than biomass-based. & If possible, derive
bag limit discarding rate using observations of catch rate and discard
rate \\
Knife edge weight / length at age of 1 at age 2+ & If possible, derive a
(vague) model for size (age) selectivity given size (age)
availability \\
Density time series CV is 10\% & Include density index series to allow
for testing of density based opening - closure rules (i.e.~based on sea
cucumbers per meter of shoreline) \\
Numbers time series CV is 10\% & Dynamic selectivity calculations (might
be a better example in urchin / hake) \\
Model includes a k (growth rate) for fecundity at age (cubic on inplied
growht) between 0.3 and 0.5 & Add correct coords to OM objects by QMA \\
& Check management intervals in OM \\
& Can evalute survey frequency (2 - 4 years currently) \\
& What density per subarea is indicative of overfishing~ \\
& How precautionary are current management approaches? \\
& What is the conservation benefit of no-take subareas? \\
& Is the 2.5 C/m shoreline reopening threshold important to
conservation? \\
& Which areas are currently most at risk? \\
& What is the relative performance of Control points based on unfished
density (by subarea) rather than a nominal density (e.g.~x c/m-sl) \\
& What is the relative performance of alternative control points? \\
& Subarea control points - how well do these achieve overall (summed)
LRP performance across multiple areas? \\
& 15 cm `pencil' effect (is there some other reference that is better
for fishermen?) \\
\end{longtable}

~

\paragraph{Sea Cucumber Meeting Notes
etc}\label{sea-cucumber-meeting-notes-etc}

\href{Project_Info/Draft\%20Summary\%20of\%20Sea\%20Cucumber\%20CSRF\%20Meeting.pdf}{Feb
2024 Meeting Notes (.pdf)}

~

\begin{center}\rule{0.5\linewidth}{0.5pt}\end{center}

\subsubsection{Software and Code}\label{software-and-code}

\href{https://github.com/mis-assess/csrf_hh_data}{csrf\_hh\_data GitHub
Repository}

\href{https://openMSE.com}{openMSE (MSEtool, DLMtool, SAMtool R
libraries)}

\href{https://samtool.openmse.com/reference/RCM.html}{Rapid Conditioning
Model (RCM) (Huynh 2023)}

\begin{center}\rule{0.5\linewidth}{0.5pt}\end{center}

\subsubsection{Recent Presentations}\label{recent-presentations}

\href{Presentations/Geoduck\%20v2.pdf}{Geoduck Nov 2023 (.pdf)}

\href{Presentations/Manilla\%20Clam\%20v4.pdf}{Manilla Clam Nov 2023
(.pdf)}

\href{Presentations/Cucumber\%203.pdf}{Cucumber Feb 2024 (.pdf)}

\href{Presentations/Urchin\%204.pdf}{Urchin Feb 2024 (.pdf)}

\begin{center}\rule{0.5\linewidth}{0.5pt}\end{center}

\subsubsection{References}\label{references}

\href{References/MPframework.pdf}{DFO 2021 (Anderson et al)}

\href{References/Geoduck_IFMP.pdf}{Geoduck IFMP}

\href{References/GSU_IFMP.pdf}{Green Sea Urchin IFMP}

\href{References/Sea_Cucumber_IFMP.pdf}{Sea Cucumber IFMP}

\href{References/Intertidal_Clams_IFMP.pdf}{Intertidal Clam IFMP}

\begin{center}\rule{0.5\linewidth}{0.5pt}\end{center}

\subsubsection{Acknowledgements}\label{acknowledgements}

Many thanks to Shannon Obradovich and Mackenzie Mazur for helping to
direct and manage the research project.

Special thanks to Rob Flemming for his help in providing and explaining
the various datasets and also to Dominique Bureau for providing guidance
on data interpretation and reviewing materials.

Thanks to Ken Fong for his overview and expertise on the historical
aspects of the fisheries, science and management programmes.

Many thanks also to the contribution of individuals on the
species-specific analyses:

(Technical support) Meghan Burton; Mackenzie Mazur; Kelsey Dougan;

(Managers) Amy Ganton, Brittany Myhal, Pauline Ridings, Jenny Smith,
Erin Wylie;

(Geoduck collaborators) Erin Porszt, Dominique Bureau;

(Manila clam collaborators) Alexander Dalton, Dominique Bureau, Coral
Cargill;

(Green urchin collaborators) Lyanne Curtis, Christine Hansen, Travis
Bell;

(Sea cucumber collaborators) Jill Campbell, Christine Hansen, Erin
Wylie, Travis Bell;

\begin{center}\rule{0.5\linewidth}{0.5pt}\end{center}

~~~~~~~~~~~~

\end{document}
